\documentclass[12pt]{article}
\begin{document}
\title{A Latex report of 5 Medical devices}
\maketitle{Medical Devices}

\section{ECG}
\subsection{Introduction}
Electrocardiography is the process of producing an electrocardiogram (ECG or EKG), a recording of the heart's electrical activity.It is an electrogram of the heart which is a graph of voltage versus time of the electrical activity of the heart using electrodes placed on the skin. These electrodes detect the small electrical changes that are a consequence of cardiac muscle depolarization followed by repolarization during each cardiac cycle (heartbeat). Changes in the normal ECG pattern occur in numerous cardiac abnormalities, including cardiac rhythm disturbances (such as atrial fibrillation and ventricular tachycardia, inadequate coronary artery blood flow (such as myocardial ischemia and myocardial infarction), and electrolyte disturbances (such as hypokalemia and hyperkalemia).\\Traditionally, "ECG" usually means a 12-lead ECG taken while laying down as discussed below. However, other devices can record the electrical activity of the heart such as a Holter monitor but also some models of smartwatch are capable of recording an ECG. ECG signals can be recorded in other contexts with other devices.

In a conventional 12-lead ECG, ten electrodes are placed on the patient's limbs and on the surface of the chest. The overall magnitude of the heart's electrical potential is then measured from twelve different angles ("leads") and is recorded over a period of time (usually ten seconds). In this way, the overall magnitude and direction of the heart's electrical depolarization is captured at each moment throughout the cardiac cycle.

There are three main components to an ECG: the P wave, which represents the depolarization of the atria; the QRS complex, which represents the depolarization of the ventricles; and the T wave, which represents the repolarization of the ventricles.
\subsection{How to use}
During each heartbeat, a healthy heart has an orderly progression of depolarization that starts with pacemaker cells in the sinoatrial node, spreads throughout the atrium, and passes through the atrioventricular node down into the bundle of His and into the Purkinje fibers, spreading down and to the left throughout the ventricles.[13] This orderly pattern of depolarization gives rise to the characteristic ECG tracing. To the trained clinician, an ECG conveys a large amount of information about the structure of the heart and the function of its electrical conduction system.[14] Among other things, an ECG can be used to measure the rate and rhythm of heartbeats, the size and position of the heart chambers, the presence of any damage to the heart's muscle cells or conduction system, the effects of heart drugs, and the function of implanted pacemakers.
\subsection{Medical uses}
The overall goal of performing an ECG is to obtain information about the electrical function of the heart. Medical uses for this information are varied and often need to be combined with knowledge of the structure of the heart and physical examination signs to be interpreted. Some indications for performing an ECG include the following:[citation needed]

Chest pain or suspected myocardial infarction (heart attack), such as ST elevated myocardial infarction (STEMI) or non-ST elevated myocardial infarction (NSTEMI)
Symptoms such as shortness of breath, murmurs,[18] fainting, seizures, funny turns, or arrhythmias including new onset palpitations or monitoring of known cardiac arrhythmias
Medication monitoring (e.g., drug-induced QT prolongation, Digoxin toxicity) and management of overdose (e.g., tricyclic overdose)
Electrolyte abnormalities, such as hyperkalemia
Perioperative monitoring in which any form of anesthesia is involved (e.g., monitored anesthesia care, general anesthesia). This includes preoperative assessment and intraoperative and postoperative monitoring.
Cardiac stress testing
Computed tomography angiography (CTA) and magnetic resonance angiography (MRA) of the heart (ECG is used to "gate" the scanning so that the anatomical position of the heart is steady)
Clinical cardiac electrophysiology, in which a catheter is inserted through the femoral vein and can have several electrodes along its length to record the direction of electrical activity from within the heart.
ECGs can be recorded as short intermittent tracings or continuous ECG monitoring. Continuous monitoring is used for critically ill patients, patients undergoing general anesthesia,and patients who have an infrequently occurring cardiac arrhythmia that would unlikely be seen on a conventional ten-second ECG. Continuous monitoring can be conducted by using Holter monitors, internal and external defibrillators and pacemakers, and/or biotelemetry.

\section{Magnetic Image Resonance(MRI)}
\subsection{Introduction}
Magnetic resonance imaging (MRI) is a medical imaging technique used in radiology to form pictures of the anatomy and the physiological processes of the body. MRI scanners use strong magnetic fields, magnetic field gradients, and radio waves to generate images of the organs in the body. MRI does not involve X-rays or the use of ionizing radiation, which distinguishes it from CT and PET scans. MRI is a medical application of nuclear magnetic resonance (NMR) which can also be used for imaging in other NMR applications, such as NMR spectroscopy.

MRI is widely used in hospitals and clinics for medical diagnosis, staging and follow-up of disease. Compared to CT, MRI provides better contrast in images of soft-tissues, e.g. in the brain or abdomen. However, it may be perceived as less comfortable by patients, due to the usually longer and louder measurements with the subject in a long, confining tube. Additionally, implants and other non-removable metal in the body can pose a risk and may exclude some patients from undergoing an MRI examination safely.
\subsection{History}
The history of MRI can be considered in three phases: the discovery of the fundamental physics and biological properties of nuclear magnetic resonance, the emergence of designs to accomplish imaging with MRI, and the emergence of neurologically optimized methods such as diffusion tensor tractography and functional MRI.\\
MRI was originally called NMRI (nuclear magnetic resonance imaging), but "nuclear" was dropped to avoid negative associations.Certain atomic nuclei are able to absorb radio frequency energy when placed in an external magnetic field; the resultant evolving spin polarization can induce a RF signal in a radio frequency coil and thereby be detected.In clinical and research MRI, hydrogen atoms are most often used to generate a macroscopic polarization that is detected by antennae close to the subject being examined.Hydrogen atoms are naturally abundant in humans and other biological organisms, particularly in water and fat. For this reason, most MRI scans essentially map the location of water and fat in the body. Pulses of radio waves excite the nuclear spin energy transition, and magnetic field gradients localize the polarization in space. By varying the parameters of the pulse sequence, different contrasts may be generated between tissues based on the relaxation properties of the hydrogen atoms therein.

Since its development in the 1970s and 1980s, MRI has proven to be a versatile imaging technique. While MRI is most prominently used in diagnostic medicine and biomedical research, it also may be used to form images of non-living objects, such as Mummified Humans. Diffusion MRI and Functional MRI extends the utility of MRI to capture neuronal tracts and blood flow respectively in the nervous system, in addition to detailed spatial images. The sustained increase in demand for MRI within health systems has led to concerns about cost effectiveness and overdiagnosis.
\subsection{Construction}
The major components of an MRI scanner are the main magnet, which polarizes the sample, the shim coils for correcting shifts in the homogeneity of the main magnetic field, the gradient system which is used to localize the region to be scanned and the RF system, which excites the sample and detects the resulting NMR signal. The whole system is controlled by one or more computers.

MRI requires a magnetic field that is both strong and uniform to a few parts per million across the scan volume. The field strength of the magnet is measured in teslas – and while the majority of systems operate at 1.5 T, commercial systems are available between 0.2 and 7 T. Most clinical magnets are superconducting magnets, which require liquid helium to keep them very cold. Lower field strengths can be achieved with permanent magnets, which are often used in "open" MRI scanners for claustrophobic patients.[7] Lower field strengths are also used in a portable MRI scanner approved by the FDA in 2020.[8] Recently, MRI has been demonstrated also at ultra-low fields, i.e., in the microtesla-to-millitesla range, where sufficient signal quality is made possible by prepolarization (on the order of 10–100 mT) and by measuring the Larmor precession fields at about 100 microtesla with highly sensitive superconducting quantum interference devices (SQUIDs).
\subsection{Safety}
MRI is, in general, a safe technique, although injuries may occur as a result of failed safety procedures or human error.Contraindications to MRI include most cochlear implants and cardiac pacemakers, shrapnel, and metallic foreign bodies in the eyes. Magnetic resonance imaging in pregnancy appears to be safe, at least during the second and third trimesters if done without contrast agents. Since MRI does not use any ionizing radiation, its use is generally favored in preference to CT when either modality could yield the same information. Some patients experience claustrophobia and may require sedation or shorter MRI protocols.Amplitude and rapid switching of gradient coils during image acquisition may cause peripheral nerve stimulation.

MRI uses powerful magnets and can therefore cause magnetic materials to move at great speeds, posing a projectile risk, and may cause fatal accidents.However, as millions of MRIs are performed globally each year,fatalities are extremely rare.\\Medical societies issue guidelines for when physicians should use MRI on patients and recommend against overuse. MRI can detect health problems or confirm a diagnosis, but medical societies often recommend that MRI not be the first procedure for creating a plan to diagnose or manage a patient's complaint. A common case is to use MRI to seek a cause of low back pain; the American College of Physicians, for example, recommends against this procedure as unlikely to result in a positive outcome for the patient.
\section{Ergometer}
\subsection{Introduction}
An ergometer is an exercise machine that tests the exertion exhibited by certain muscles or that keeps track of how much of a particular exercise has been done. It can also refer to certain exercise machines that allow someone to perform cardio exercises using the arms and legs while remaining in a stationary position.\\The benefits of using ergometers are the same as with using any other cardio workout equipment. The heart, lungs, bones and muscles all benefit from the use of ergometers. Stress release, weight loss, toning leg and arm muscles are a few examples of what an ergometer can do for you. Ergometers have been become popular in physical therapy for people who have physical disabilities like individuals that are confined to a wheelchair or those who are not physically capable of performing more conventional workouts.
\subsection{Types of Ergometer}
There are different types of ergometer machines, some of which only work out the arms, others that work out the legs, and some that do both at the same time.\\

Upper Extremity Ergometer\\

If you’ve ever seen a physical therapist for an upper extremity injury such as, rotator cuff surgery or shoulder bursitis, proximal humerus fracture, radial head or elbow fracture, Colles or Smiths fracture, tennis elbow or golfer’s elbow, clavicle fracture, shoulder dislocation or labrum tear, chances are your physical therapist had you use an upper extremity ergometer to measure how much work your upper body muscles are doing. This is done by grasping the the handles and turning them in a circular motion. People that can also benefit from the upper extremity ergometer are individuals that can’t use their legs and still want to get a cardiovascular workout. Upper extremity machines have different settings to increase or decrease resistance. They also have an adjustable seat, and many allow you to use the machine while standing up. Examples of these are the SciFit Pro1000, SciFit Pro1 and Technogym Excite UBE.\\

Lower Extremity Ergometer\\

A lower extremity ergometer is pretty much the opposite of a upper extremity ergometer. A lower extremity ergometer focuses on the lower body muscles like the gluteus maximus and the hamstrings through the use of pedaling while increasing or decreasing resistance. Physical therapists usually have their patients exercise on a lower extremity ergometer if they have sustained a lower body injury. Lower extremity ergometer helps patients regain range of motion and strength in their lower body. People that have lost the use of their arms can also benefit from the lower extremity ergometer to get a cardiovascular workout. Lower extremity ergometers usually have handle bars for the user to hold on to while using the machine.\\

Dual Extremity Ergometer\\

Dual extremity ergometers work the upper and lower body through resistance by using your legs and arms in the same way the upper extremity and lower extremity ergometers are used. People that have trouble standing can benefit from this machine since it is a total body workout that can be done while being seated.
\subsection{Purpose}
The purpose of a cycle ergometer is chiefly to measure cardiac performance—specifically, maximum heart rate and oxygen uptake. Cycle ergometers are used as part of standardized testing procedures that evaluate the tested individual against set performance criteria and that occur according within a set of defined parameters.

Specific tests that utilize cycle ergometers include the YMCA Sub Max Cycle Ergometer Test and the simpler-to-perform Astrand-Rhyming (A-R) Cycle Ergometer Test.Other forms of ergometry that do not use stationary bicycles include treadmill-based ergometric testing. The type of test selected for use depends on the goal of the test, as well as whether an individual can comfortably perform a specific test. Individuals with limited range of motion at the knees may have difficulty rotating bicycle pedals and therefore would require an alternative form of testing.
\section{Stethoscope}
\subsection{Introduction}
he stethoscope is an acoustic medical device for auscultation, or listening to internal sounds of an animal or human body. It typically has a small disc-shaped resonator that is placed against the skin, and one or two tubes connected to two earpieces. A stethoscope can be used to listen to the sounds made by the heart, lungs or intestines, as well as blood flow in arteries and veins. In combination with a manual sphygmomanometer, it is commonly used when measuring blood pressure.

Less commonly, "mechanic's stethoscopes", equipped with rod shaped chestpieces, are used to listen to internal sounds made by machines (for example, sounds and vibrations emitted by worn ball bearings), such as diagnosing a malfunctioning automobile engine by listening to the sounds of its internal parts. Stethoscopes can also be used to check scientific vacuum chambers for leaks and for various other small-scale acoustic monitoring tasks.

A stethoscope that intensifies auscultatory sounds is called a phonendoscope.
\subsection{History}
The stethoscope was invented in France in 1816 by René Laennec at the Necker-Enfants Malades Hospital in Paris.It consisted of a wooden tube and was monaural. Laennec invented the stethoscope because he was not comfortable placing his ear directly onto a woman's chest to listen to her heart.He observed that a rolled piece of paper, placed between the individual's chest and his ear, could amplify heart sounds without requiring physical contact.Laennec's device was similar to the common ear trumpet, a historical form of hearing aid; indeed, his invention was almost indistinguishable in structure and function from the trumpet, which was commonly called a "microphone". Laennec called his device the "stethoscope"(stetho- + -scope, "chest scope"), and he called its use "mediate auscultation", because it was auscultation with a tool intermediate between the individual's body and the physician's ear. (Today the word auscultation denotes all such listening, mediate or not.) The first flexible stethoscope of any sort may have been a binaural instrument with articulated joints not very clearly described in 1829.In 1840, Golding Bird described a stethoscope he had been using with a flexible tube. Bird was the first to publish a description of such a stethoscope, but he noted in his paper the prior existence of an earlier design (which he thought was of little utility) which he described as the snake ear trumpet. Bird's stethoscope had a single earpiece.
\subsection{Types of Stethoscope}
Acoustic

Acoustic stethoscopes operate on the transmission of sound from the chest piece, via air-filled hollow tubes, to the listener's ears. The chestpiece usually consists of two sides that can be placed against the patient for sensing sound: a diaphragm (plastic disc) or bell (hollow cup). If the diaphragm is placed on the patient, body sounds vibrate the diaphragm, creating acoustic pressure waves which travel up the tubing to the listener's ears. If the bell is placed on the patient, the vibrations of the skin directly produce acoustic pressure waves traveling up to the listener's ears. The bell transmits low frequency sounds, while the diaphragm transmits higher frequency sounds.\\
Electronic\\
An electronic stethoscope (or stethophone) overcomes the low sound levels by electronically amplifying body sounds. However, amplification of stethoscope contact artifacts, and component cutoffs (frequency response thresholds of electronic stethoscope microphones, pre-amps, amps, and speakers) limit electronically amplified stethoscopes' overall utility by amplifying mid-range sounds, while simultaneously attenuating high- and low- frequency range sounds. Currently, a number of companies offer electronic stethoscopes. Electronic stethoscopes require conversion of acoustic sound waves to electrical signals which can then be amplified and processed for optimal listening. Unlike acoustic stethoscopes, which are all based on the same physics, transducers in electronic stethoscopes vary widely. The simplest and least effective method of sound detection is achieved by placing a microphone in the chestpiece.\\

A 3D-printed stethoscope\\
It is an open-source medical device meant for auscultation and manufactured via means of 3D printing. The 3D stethoscope was developed by Dr. Tarek Loubani and a team of medical and technology specialists. The 3D-stethoscope was developed as part of the Glia project, and its design is open source from the outset. 

\section{Treadmill}
\subsection{Introduction}
A treadmill is a device generally used for walking, running, or climbing while staying in the same place. Treadmills were introduced before the development of powered machines to harness the power of animals or humans to do work, often a type of mill operated by a person or animal treading the steps of a treadwheel to grind grain. In later times, treadmills were used as punishment devices for people sentenced to hard labor in prisons. The terms treadmill and treadwheel were used interchangeably for the power and punishment mechanisms.

More recently, treadmills have instead been used as exercise machines for running or walking in one place. Rather than the user powering a mill, the device provides a moving platform with a wide conveyor belt driven by an electric motor or a flywheel. The belt moves to the rear, requiring the user to walk or run at a speed matching the belt. The rate at which the belt moves is the rate of walking or running. Thus, the speed of running may be controlled and measured.
\subsection{History} 
William Staub, a mechanical engineer, developed the first consumer treadmill for home use.[2] Staub developed his treadmill after reading the 1968 book, Aerobics by Kenneth H. Cooper. Cooper's book noted that individuals who ran for eight minutes four to five times a week would be in better physical condition. Staub noticed that there were no affordable household treadmills at the time and decided to develop one for his own use during the late 1960s. He called his first treadmill the PaceMaster 600. Once finished, Staub sent his prototype treadmill to Cooper, who found the machine's first customers, including sellers of fitness equipment.
Staub began producing the first home treadmills at his plant in Clifton, New Jersey, before moving production to Little Falls, New Jersey.
\subsection{Advantages of Treadmill}
I-Enable the user to adhere to an indoor exercise regime irrespective of the weather.\\
II-Cushioned tread can provide slightly lower impact training than running on outdoor surfaces. Although cushioned belts have mostly been phased out and cushioned replacement belts may be hard to find, many treadmills have rubber or urethane deck elastomers (cushions) which are superior in cushioning and last longer than cushioned belts. For a time, banana shaped flexible decks were available which were among the very best for cushioning and were priced at a mid-range level, but these are no longer being sold, perhaps because of the increased manufacturing cost of making flexible decks. Cushioned belts do not last as long as regular belts due to their construction from weaker materials. For calorie burning, incline can be used to significantly reduce impact for a given rate of energy use.\\
III-Incline setting can allow for consistent "uphill" training that is not possible when relying on natural features.\\
IV-Rate settings force a consistent pace.\\
V-Some treadmills have programmes so that the user can simulate terrains, e.g. rolling hills, to provide accurate, programmed, exercise periods.\\
VI-The user can watch TV whilst using the machine, thus preventing TV watching from being a sedentary activity.\\
VII-User progress such as distance, calories burned, and heart rate can be tracked.
\subsection{Disadvantages of Treadmill}
As a cardiovascular exercise:

I-Some treadmill runners develop poor running habits that become apparent when they return to outdoor running. In particular a short, upright, bouncy gait may result from having no wind resistance and trying to avoid kicking the motor covering with the front of the foot.\\
II-Imposes a strict pace on runners, giving an unnatural feel to running which can cause a runner to lose balance.\\
III-Treadmill running is not specific to any sport, i.e., there is no competitive sport that actually utilizes treadmill running. For example, a competitive runner would be far better off running outdoors through space since it is more specific and realistic to their event.\\
IV-There are differences in temporal and angular kinematics which should be considered when treadmills are used within a rehabilitation program.\\
As an indoor activity:

I-Many users find treadmills monotonous and lose interest after a period.\\
II-Treadmills do not offer the psychological satisfaction some runners get from running in new locations away from the distractions of home.\\
As a machine:

I-May cause personal injury if not used properly. Of particular concern are children who reach into the treadmill belt while it is running and suffer severe friction burns that in the worst case may require multiple skin grafts and result in lasting disability.[14] Injury to children can be avoided by removing the safety key when the treadmill is not in use, without which, the treadmill belt will not start.\\
II-Costs of purchase, electrical costs, and possible repair are significantly greater than those of running outside.\\
III-Takes up space in homes.\\

\end{document}

